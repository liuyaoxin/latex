\documentclass{article}
\usepackage[adobefonts]{ctex} %支持中文就靠这一行了
\usepackage{fontspec}
\usepackage{tikz}
\usepackage[active,xetex,tightpage]{preview} %这一行有改动,将pdftex换成xetex

\PreviewEnvironment[]{tikzpicture}
\PreviewEnvironment[]{pgfpicture}
\DeclareSymbolFont{symbolsb}{OMS}{cmsy}{m}{n}
\SetSymbolFont{symbolsb}{bold}{OMS}{cmsy}{b}{n}
\DeclareSymbolFontAlphabet{\mathcal}{symbolsb}

\usepackage{xcolor}
\usepackage{pgfplots}
\usepackage{tikz}

\newcommand\YaHei{\fontspec{微软雅黑}}

%
% Define bar chart colors
%
\definecolor{bblue}{HTML}{4F81BD}
\definecolor{rred}{HTML}{C0504D}
\definecolor{ggreen}{HTML}{9BBB59}
\definecolor{ppurple}{HTML}{9F4C7C}

\pgfplotsset
{
	tick label style = {font=\YaHei},
	xticklabel style = {font=\YaHei},
	legend style     = {font=\YaHei,cells={anchor=west}},
}

\begin{document}

% ybar控制同一组内bar之间的微小间隔
% ymin控制Y轴的最小数字,ymax控制Y轴的最大数字
\begin{tikzpicture}
    \begin{axis}[
        major x tick style = transparent,
		width  = 0.85*\textwidth,
        height = 8cm,        
        ybar   = 5*\pgflinewidth,
        bar width   = 24pt,
        ymajorgrids = true,
		title  = {\YaHei\bfseries 苹果手机销售情况},
		ymin   = 0,
		ymax   = 100,
		xtick  = data,		
        symbolic x coords = {2006,2007,2008,2009,2010,2011,2012},
        scaled y ticks    = false,        
        legend cell align = left,
		legend pos = north east,
		nodes near coords,% 在柱子顶部显示实际数字
        tick align = inside,
		bar shift = 0pt,
		enlarge x limits = {abs = 0.5cm}, % The distance between the center of the first bar and the left edge
		enlarge y limits = false,
    ]
		\legend{\bfseries{亿元}}
        \addplot[style={bblue,fill=bblue,mark=none}]
            coordinates {(2006, 10)(2007, 60)(2008, 70)(2009, 88)(2010, 44)(2011, 47)(2012, 54)};
			
    \end{axis}
\end{tikzpicture}

\end{document}