\documentclass[border=12pt,pstricks]{standalone}
\usepackage[adobefonts]{ctex} %支持中文就靠这一行了
\usepackage{pstricks-add}
\usepackage{fontspec}
\newcommand\YaHei{\fontspec{微软雅黑}}

%
% 注意这里定义的宏,决定圆圈的大小,从原点画圆,移动的任务留给其他命令。只负责画圆。
% 注意这里画了两个重叠的圆,下面的圆是浅灰色,上面的圆是分块的颜色,构成一个圆环。
% \pscircle*画出实心圆,\pscircle画空心圆。
% put就是移动当前原点的意思。深刻理解当前原点的概念。原点在当前待输入字符方框的左下角。
% \nput的第一个参数是参考方向『refangle』。
% 参考方向用角度表示。参考方向的作用:画出的结果图形会沿着参考方向平移一段距离,通常更美观。
% 注意参考方向决定平移而不是旋转。换句话说,盒子始终不会倾斜。
% 参考方向就是相对于A4纸张构成的直角坐标系的旋转角度。
% 小圆的圆心位于大圆半径的1.25倍处。半径的延长段的终点就是小圆的圆心。小圆根据指定的参考方向的平移。
% 参考方向就是角平分线与X+轴构成的角度。计算公式:块百分比 × 180 + 前面所有块百分比之和 × 360。
%
\def\CPut#1#2{\pscircle*[linecolor=black!15](0,0){5.5mm}%
	\pscircle[linecolor=#1!70](0,0){5.5mm}\rput(0,0){\tiny\shortstack{#2}}} % 圆圈中的文字为tiny,\shortstack允许圆内文字换行

\begin{document}
	\begin{pspicture}(-3.5,-3.5)(3.5,3.5)
		% 下面这一行设置饼图面积。
		% {25,35,25,15}表示有四块,同时指出每一块的百分比。
		% 接下来的{4}表示outraged pie(切开的那块饼,可以有多块)的块号(从1开始编号,4就代表第4块)。
		% 接下来的{2}表示圆的半径。
		% chartNodeO调整直线的长度
		\psChart[userColor={blue!70,orange!70,green!70,purple!70},chartSep=8pt,chartNodeO=1.30]{25,35,25,15}{4}{2}
		%\pscircle*{0.5} % 画核心小圆圈。默认单位为1cm。圆心位于原点,可以省略参数。
		\pscircle*[linecolor=white]{0.5} % 画白色的实心圆,从而把饼图变成圆环
		\psset{nodesepA=5pt,nodesepB=-10pt}
		\YaHei\bfseries % 通篇设置为雅黑粗体无压力!

		% \ncline的作用是在两个node之间划一条直线,大圆是一个node,小圆也是一个node。
		% \Cput的作用是画圆和文字。
		% \nput的第一个参数指定圆圈
		% 说明圆圈中的文字最多5个字,否则会突破边界,难看!
		\ncline{psChartO1}{psChart1}
		\nput{45}{psChartO1}
		{\CPut{blue}{中国共产党 \\
		{25\,\%}}}
			
		\ncline{psChartO2}{psChart2}
		\nput{153}{psChartO2}
		{\CPut{orange}{九三学社 \\
		{35\,\%}}}
		
		\ncline{psChartO3}{psChart3}
		\nput{261}{psChartO3}
		{\CPut{green}{中国致公党 \\
		{25\,\%}}}
		
		\ncline{psChartO4}{psChart4}
		\nput{333}{psChartO4}
		{\CPut{purple}{台湾青盟 \\
		{15\,\%}}}

		% 下面这一行控制圆圈内部的文字
		% \rput(psChartI1){Taxes}\rput(psChartI2){Rent}\rput(psChartI3){Bills}\rput(psChartI4){Car}		
		% \rput(0,0){\tiny\white{中国饼图}}
		% \rput(0,0){\tiny{中国饼图}}
		\rput(0,3.0){\YaHei\bfseries\footnotesize{中国Microsoft饼图}} % 添加标题
	\end{pspicture}
\end{document}