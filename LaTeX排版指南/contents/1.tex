\section{\LaTeX{}排版的基本观念}

\Warning{}司创始人Motolola腾讯公司创始人。

\Warning{}公司创始人Motolola腾讯公司创始人。

\Warning{}腾讯公司创始人Motolola腾讯公司创始人。

\subsection{排版和创作是相互正交的两种智力活动}
创作的时候不应该考虑排版。创作过程中,只需要按照最简单的\emph{顺序结构和层次结构}去安排章节,不用去关心章节编号、字体样式等与创作内容无关的细节性问题。

\LaTeX

排版的时候也不应该考虑创作的具体内容。只需要考虑如何自动编号章节和图表;如何选择合适的字体;如何自动生成目录等等。

\begingroup
腾讯公司创始人Motolola Motolola腾讯公司创始人

腾讯公司创始人\bfseries{Motolola Motolola}腾讯公司创始人

腾讯公司创始人\itshape{Motolola Motolola}腾讯公司创始人

腾讯公司创始人\sffamily{Motolola Motolola}腾讯公司创始人

腾讯公司创始人\ttfamily{Motolola Motolola}腾讯公司创始人

腾讯公司创始人\bfseries\itshape{Motolola Motolola}腾讯公司创始人

腾讯公司创始人\itshape\sffamily{Motolola Motolola}腾讯公司创始人

腾讯公司创始人\bfseries\itshape\sffamily{Motolola Motolola}腾讯公司创始人

腾讯公司创始人\bfseries\sffamily{Motolola Motolola}腾讯公司创始人

腾讯公司创始人\sffamily{Motolola Motolola}腾讯公司创始人

腾讯公司创始人\ttfamily{Motolola Motolola}腾讯公司创始人
\endgroup

\subsection{用语义性指令表达排版要求}
\label{subsec:semantics}

创作过程中,不应显式地嵌入排版样式指令。如果需要强调一段文字,不要这样做:

\begin{lstlisting}[language=TeX]
这是一段\bdseries\itshape{需要强调}的文字。% 用粗体加斜体表示强调
\end{lstlisting}

应该这样做:

\begin{lstlisting}[language=TeX]
这是一段\emph{需要强调}的文字。
\end{lstlisting}

\textbackslash{}emph就是emphasis的意思;至于怎么表示强调,是排版阶段决定的事情。这样可以准确传递作者的原始意图,可以确保全文具有统一的风格。

\subsection{空格和空行的特殊行为}

在\LaTeX{}中:

\begin{Itemize}
\item {一个空格和多个空格等价,视为一个空格。}
\item {一个空行和多个空行等价,视为一个空行。}
\item {在一行文字前面的任意多个空格都将被忽略,视为无空格。}
\item {在一个段落前面的任意多个空行都将被忽略,视为无空行。}
\item {\textbackslash{}hspace可用于增加字间距。}
\item {\textbackslash{}vspace可用于增加行间距。}
\item {\textbackslash{}quad和\textbackslash{}qquad是增加字间距的简便方法。}
\end{Itemize}

\subsection{\LaTeX{}的保留字符}

下面的这些字符是\LaTeX{}的保留字符:

\# \qquad \$ \qquad \% \qquad \& \qquad \_ \qquad \{ \qquad \} \qquad \textasciicircum{} \qquad \textasciitilde{} \qquad \textbackslash{}

大多数保留字符的输出方法是在其前面加上\textbackslash{}:

\textbackslash{}\# \qquad \textbackslash{}\$ \qquad \textbackslash{}\% \qquad \textbackslash{}\& \qquad \textbackslash{}\_ \qquad \textbackslash{}\{ \qquad \textbackslash{}\}

\emph{注意:\textasciicircum{}、\textasciitilde{}和\textbackslash{}的输出方法比较特殊:}

\textasciicircum{}的输出命令:\textbackslash{}textasciicircum\{\}

\textasciitilde{}的输出命令:\textbackslash{}textasciitilde\{\}

\textbackslash{}的输出命令:\textbackslash{}textbackslash\{\}

\subsection{特殊符号的输出方法}

特殊符号可以通过命令输出。例如:

\textbackslash{}LaTeXe\{\}输出:\LaTeXe{}

\textbackslash{}copyright\{\}输出:\copyright{}。

\subsection{中文和English Text混排效果}
\label{subsec:chinese_mix_english}

默认情况下,中英文混排将会采用同一种字体输出中文和English,有些字体中文好看,But the English text looks very ugly. 因此,必须找到一种中英文输出都很漂亮的字体。这将从根本上决定混排效果的好坏。

目前,我们认为\emph{微软雅黑、Adobe楷体、Times New Roman}是最好的。其中\emph{微软雅黑}是主力字体。\emph{Adobe楷体}用于中文引文,\emph{Times New Roman}用于英文引文。至此,字体的完美组合形成了。三种字体打遍天下。

\subsection{那些难忘的坑}

由于设计失误,导致newenvironment的尾块『相对于首块』不能包含\#参数,否则编译不过。

不能把命令名字拼写错误,否则稀奇古怪的错误。