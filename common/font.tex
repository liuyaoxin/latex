%
% 不要写错Adobe的四种字体名称:Adobe Song Std,Adobe Fangsong Std,Adobe Heiti Std,Adobe Kaiti Std
% 
% 大段英文优先选用Arial,主要是和Adobe黑体的风格相互协调。
%
% 主力中文字体2类:Adobe(4种)
% 主力英文字体3类:Arial, Times New Roman, Courier New
%
% CTex定义了四种常见中文字体:\heiti, \songti, \fangsong, \kaishu。
%
\newcommand*{\Arial}{\fontspec{Arial}} % Best Sans-Serif Font in pure English Environment
\newcommand*{\TimesNewRoman}{\fontspec{Times New Roman}} % Best Serif font in pure English environment
\newcommand*{\CodeFont}{\fontspec{Courier New}} % 最佳代码显示字体, Best monospace font
\newcommand*{\CourierNew}{\fontspec{Courier New}} % 最佳中英文混排中的英文字体
\newcommand*{\kaiti}{\kaishu}

% 注意:微软雅黑不适合显示大段英文字体,主要是其标点符号是中文风格,
% 用微软雅黑显示一段包含标点符号的英文段落会极其难看。
%
% 显示大段英文,还是临时指定特定的英文显示字体比较好。例如TimesNewRoman。
% 微软雅黑最大的优点在于中英文混排环境下的英文效果绝佳。
%
% 下面六条命令极其重要。全局性配置字体。而且是中英文分别配置。
% 中英文分别配置原理:setCJKmainfont会自动判断一个字符是否为中日韩字符。
% 如果是,就调用指定的字体,否则就调用系统原来设定的字体。
% 放在后面的命令更加优先。
%
% 中英文混排如何任意地分别指定中英文字体?
% \ChineseFont\EnglishFontSettings{Text}
%
% 黑体其实就是中文的粗体。粗体的作用有两个:1、显示标题;2、强调。
% 斜体的作用:表示引用。
% Adobe Heiti依然是最好的中文粗体。微软雅黑最大的优点在于中英文混排环境下的英文效果绝佳。
% 因此,我们的字体策略:
% 1、Heiti显示标题和强调;(这一步由 \setCJK*font 控制)
% 2、Fangsong显示正文;(这一步由 \setCJK*font 控制)
% 3、用Song表示引用部分字体;(这一步由 \setCJK*font 控制)
% 4、如果其中夹杂有英文,则自动以微软雅黑显示英文部分。(这一步由 \set*font 控制)
% 5、Arial可以完全替代微软雅黑。
%
% 英文字体设置以下面为准。在后面的导言区可以重新定义这些字体设置,以适合纯英文的不同需要。

\ifnum \value{IsEnglishArticle}>0
\setmainfont{Arial} % 罗马字体,有衬线。Courier New的笔划很细,与仿宋协调。
\setsansfont{Arial} % 无衬线
\setmonofont{Courier New} % 等宽
\else
\setmainfont{Courier New} % 罗马字体,有衬线。Courier New的笔划很细,与仿宋协调。
\setsansfont{Arial} % 无衬线
\setmonofont{Courier New} % 等宽
\fi

% 中文字体设置以下面为准。
% 注意:加粗时更换字体为黑体,斜体对中文无效
%
% 衬线字体。仿宋本质上是衬线字体。粗体用黑体代替,斜体用楷体代替。
\setCJKmainfont[BoldFont=Adobe Heiti Std,ItalicFont=Adobe Kaiti Std]{Adobe Fangsong Std} 
% 无衬线字体。黑体本质上是无衬线字体。黑体加粗或者加斜都保持原状不变。
\setCJKsansfont{Adobe Heiti Std}
% 等宽字体。中文本质上是等宽字体,默认继续使用仿宋。粗体用黑体代替,斜体用楷体代替。
\setCJKmonofont{Adobe Fangsong Std}

% 下面是非常重要的中文字体设定技巧。只有这样提前设定,才可以在文档任意位置指定这里设定的字体。直接用fontspec对中文无效。
\setCJKfamilyfont{YaHeiFamily}{微软雅黑}
\newcommand*{\YaHei}{\CJKfamily{YaHeiFamily}}

\newcommand\DefineChineseFont[2]{	
	\setCJKfamilyfont{Family_#1}{#2}
	\expandafter\newcommand\csname #1\endcsname{\CJKfamily{Family_#1}}
}

% e.g. \DefineChineseFont{FangZhengZhongSong}{方正中宋}