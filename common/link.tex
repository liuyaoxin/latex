% 超链接格式
\hypersetup
{
	colorlinks=true,
	linkcolor=red,
	urlcolor=blue,
	citecolor=red,
	bookmarksopen=false,
	bookmarksnumbered=true,
	pdfborder=0 0 0
}

% 最佳脚注格式
\makeatletter
\newlength{\fnBreite}
\renewcommand{\@makefntext}[1]
{
	\settowidth{\fnBreite}{\footnotesize\@thefnmark.i}
	\protect\footnotesize\upshape
	\setlength{\@tempdima}{\columnwidth}\addtolength{\@tempdima}{-\fnBreite}
	\makebox[\fnBreite][l]{\@thefnmark \phantom{}}
	\parbox[t]{\@tempdima}{\everypar{\hspace*{1em}}\hspace*{-1em}\upshape\fangsong{#1}
}}
\makeatother
\renewcommand\thefootnote{\ding{\numexpr171+\value{footnote}}} % 定义脚注的圆圈符号
\renewcommand\footnotesep{1.5em} % 各条脚注段落之间的距离

\newcommand\Reference[1]{\textsuperscript{[\ref{#1}]}} % 设置脚注

\newcommand\RefLRS[1]{\hspace{0.5em}\ref{#1}\hspace{0.5em}} % 请参考 1.6 。
\newcommand\RefLS[1]{\hspace{0.5em}\ref{#1}} % 请参考 1.6。
\newcommand\RefRS[1]{\ref{#1}\hspace{0.5em}} % 请参考1.6 。章节 1.6显示:
\newcommand\Ref[1]{\ref{#1}} % 请参考1.6。

\newcommand\PageRefLRS[1]{\hspace{0.5em}\pageref{#1}\hspace{0.5em}} % 请参考第 3 页。
\newcommand\PageRefLS[1]{\hspace{0.5em}\pageref{#1}} % 请参考第 3页。
\newcommand\PageRefRS[1]{\pageref{#1}\hspace{0.5em}} % 请参考第3 页。
\newcommand\PageRef[1]{\pageref{#1}} % 请参考第3页。

\newcommand\UrlRefLRS[2]{\href{#1}{\hspace{0.5em}\bfseries{#2}\hspace{0.5em}}} % 网页超链接
\newcommand\UrlRefLS[2]{\href{#1}{\hspace{0.5em}\bfseries{#2}}} % 网页超链接
\newcommand\UrlRefRS[2]{\href{#1}{\bfseries{#2}\hspace{0.5em}}} % 网页超链接
\newcommand\UrlRef[2]{\href{#1}{\bfseries{#2}}} % 网页超链接

\newcommand\LinkRefLRS[2]{\hyperref[#1]{\hspace{0.5em}\bfseries{#2}\hspace{0.5em}}} % 文章内部超链接
\newcommand\LinkRefLS[2]{\hyperref[#1]{\hspace{0.5em}\bfseries{#2}}} % 文章内部超链接
\newcommand\LinkRefRS[2]{\hyperref[#1]{\bfseries{#2}\hspace{0.5em}}} % 文章内部超链接
\newcommand\LinkRef[2]{\hyperref[#1]{\bfseries{#2}}} % 文章内部超链接