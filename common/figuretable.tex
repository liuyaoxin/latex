% 设置图表显示格式
\renewcommand{\thefigure}{\thesection-\arabic{figure}}
\renewcommand{\thetable}{\thesection-\arabic{table}}
\DeclareCaptionLabelFormat{FigureLabel}{\sffamily\bfseries\footnotesize{#1\hspace{0.5em}#2}}
\DeclareCaptionLabelFormat{TableLabel}{\sffamily\bfseries\footnotesize{\raisebox{-1ex}{#1\hspace{0.5em}#2}}}

\ifnum \value{IsEnglishArticle}>0
    \captionsetup[figure]{labelformat=FigureLabel,skip=1em,labelsep=none,name={Figure}}
	\captionsetup[table]{labelformat=TableLabel,skip=1em,labelsep=none,name={Table}}
\else
	\captionsetup[figure]{labelformat=FigureLabel,skip=1em,labelsep=none,name={图}}
	\captionsetup[table]{labelformat=TableLabel,skip=1em,labelsep=none,name={表}}
\fi

% 唯一的参数代表Caption。Caption为空的情况下,不出现冒号
% \ifstrempty 是非常棒的条件测试工具!
\NewDocumentEnvironment{Figure}{m}
	{\begin{figure}[htp]\centering}
	{
		\ifstrempty{#1}
			{\caption{#1}}
			{\caption{\footnotesize\hspace{1em}#1}}
	\end{figure}
	}

% 唯一的参数代表顶部Caption。Caption不可为空。
\NewDocumentEnvironment{Table}{m}
	{\begin{table}[!h]
		\centering
		\captionsetup{font={bf,sf},skip=0.5ex}
		\caption*{{\large{#1}}}
	}
	{
		\caption{}
	\end{table}
	}
	
% 设置三线式表格
\setlength\heavyrulewidth{1.5pt} % 粗线宽度
\setlength\aboverulesep{0.5ex} % 线前距离
\setlength\belowrulesep{1ex} % 线后距离

% 使每一个Section的计数器自动清零,这样图表的编号就以section为编号空间。
\counterwithin{figure}{section}
\counterwithin{table}{section}