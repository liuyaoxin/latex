\documentclass[a4paper]{article}

\newcommand{\DefineFlag}[2]{\newcounter{#1}\setcounter{#1}{#2}}
\newcommand{\SetFlag}[2]{\setcounter{#1}{#2}}

\DefineFlag{IsEnglishArticle}{0}
\SetFlag{IsEnglishArticle}{0}
\SetFlag{HasAbstract}{0}
\SetFlag{HasToc}{0}
\SetFlag{HasReferences}{0}

\newcommand\TitleName{常用字体效果}



%%%%%%%%%%%%%%%%%%%%%%%%%%
%                        %
%     Stable Section     %
%                        %
%%%%%%%%%%%%%%%%%%%%%%%%%%
\ifnum \value{IsEnglishArticle}>0
	\newcommand\ReferencesName{References}
\else
	\newcommand\ReferencesName{参考文献}
\fi

\newcommand\AuthorName{yaoxinliu@tencent.com}
\newcommand\DateLabel{\the\year{}-\twodigit{\the\month}-\twodigit{\the\day}}

%
% 不要写错Adobe的四种字体名称:Adobe Song Std,Adobe Fangsong Std,Adobe Heiti Std,Adobe Kaiti Std
% 
% 大段英文优先选用Arial,主要是和Adobe黑体的风格相互协调。
%
% 主力中文字体2类:Adobe(4种)
% 主力英文字体3类:Arial, Times New Roman, Courier New
%
% CTex定义了四种常见中文字体:\heiti, \songti, \fangsong, \kaishu。
%
\newcommand*{\Arial}{\fontspec{Arial}} % Best Sans-Serif Font in pure English Environment
\newcommand*{\TimesNewRoman}{\fontspec{Times New Roman}} % Best Serif font in pure English environment
\newcommand*{\CodeFont}{\fontspec{Courier New}} % 最佳代码显示字体, Best monospace font
\newcommand*{\CourierNew}{\fontspec{Courier New}} % 最佳中英文混排中的英文字体
\newcommand*{\kaiti}{\kaishu}

% 注意:微软雅黑不适合显示大段英文字体,主要是其标点符号是中文风格,
% 用微软雅黑显示一段包含标点符号的英文段落会极其难看。
%
% 显示大段英文,还是临时指定特定的英文显示字体比较好。例如TimesNewRoman。
% 微软雅黑最大的优点在于中英文混排环境下的英文效果绝佳。
%
% 下面六条命令极其重要。全局性配置字体。而且是中英文分别配置。
% 中英文分别配置原理:setCJKmainfont会自动判断一个字符是否为中日韩字符。
% 如果是,就调用指定的字体,否则就调用系统原来设定的字体。
% 放在后面的命令更加优先。
%
% 中英文混排如何任意地分别指定中英文字体?
% \ChineseFont\EnglishFontSettings{Text}
%
% 黑体其实就是中文的粗体。粗体的作用有两个:1、显示标题;2、强调。
% 斜体的作用:表示引用。
% Adobe Heiti依然是最好的中文粗体。微软雅黑最大的优点在于中英文混排环境下的英文效果绝佳。
% 因此,我们的字体策略:
% 1、Heiti显示标题和强调;(这一步由 \setCJK*font 控制)
% 2、Fangsong显示正文;(这一步由 \setCJK*font 控制)
% 3、用Song表示引用部分字体;(这一步由 \setCJK*font 控制)
% 4、如果其中夹杂有英文,则自动以微软雅黑显示英文部分。(这一步由 \set*font 控制)
% 5、Arial可以完全替代微软雅黑。
%
% 英文字体设置以下面为准。在后面的导言区可以重新定义这些字体设置,以适合纯英文的不同需要。

\ifnum \value{IsEnglishArticle}>0
\setmainfont{Arial} % 罗马字体,有衬线。Courier New的笔划很细,与仿宋协调。
\setsansfont{Arial} % 无衬线
\setmonofont{Courier New} % 等宽
\else
\setmainfont{Courier New} % 罗马字体,有衬线。Courier New的笔划很细,与仿宋协调。
\setsansfont{Arial} % 无衬线
\setmonofont{Courier New} % 等宽
\fi

% 中文字体设置以下面为准。
% 注意:加粗时更换字体为黑体,斜体对中文无效
%
% 衬线字体。仿宋本质上是衬线字体。粗体用黑体代替,斜体用楷体代替。
\setCJKmainfont[BoldFont=Adobe Heiti Std,ItalicFont=Adobe Kaiti Std]{Adobe Fangsong Std} 
% 无衬线字体。黑体本质上是无衬线字体。黑体加粗或者加斜都保持原状不变。
\setCJKsansfont{Adobe Heiti Std}
% 等宽字体。中文本质上是等宽字体,默认继续使用仿宋。粗体用黑体代替,斜体用楷体代替。
\setCJKmonofont{Adobe Fangsong Std}

% 下面是非常重要的中文字体设定技巧。只有这样提前设定,才可以在文档任意位置指定这里设定的字体。直接用fontspec对中文无效。
\setCJKfamilyfont{YaHeiFamily}{微软雅黑}
\newcommand*{\YaHei}{\CJKfamily{YaHeiFamily}}

\newcommand\DefineChineseFont[2]{	
	\setCJKfamilyfont{Family_#1}{#2}
	\expandafter\newcommand\csname #1\endcsname{\CJKfamily{Family_#1}}
}

% e.g. \DefineChineseFont{FangZhengZhongSong}{方正中宋}
\usepackage[a4paper,centering,scale=0.75]{geometry}
\usepackage[linktocpage=true]{hyperref}
\usepackage{hologo} % logo symbols like \XeTeX, etc
\usepackage{mflogo} % metafont, metapost
\usepackage{metalogo}
\usepackage{amsmath}
\usepackage{framed}
\usepackage{lipsum}
\usepackage[framemethod=tikz]{mdframed}
\usepackage{fancybox}
\usepackage{varwidth}
\usepackage{doc}
\usepackage{tocloft}
\usepackage{titlesec}
\usepackage{pifont}
\usepackage{bbding}
\usepackage{datetime}
\usepackage{setspace}
\usepackage{amsmath}
\usepackage{listings}
\usepackage{enumitem}
\usepackage{graphicx}
\usepackage{hhline}
\usepackage{caption}
\usepackage{ifthen}
\usepackage{booktabs}
\usepackage{relsize}
\usepackage{xparse} % 避免\newenvironment莫名其妙的错误
\usepackage{xcolor}
\usepackage{etoolbox} % provide \ifstrempty
\usepackage{chngcntr} % 可以使计数器自动清零
\usepackage[indentfirst=true,leftmargin=0em,rightmargin=0em]{quoting} % 最佳引用效果

% \usepackage[os=mac/win,mackeys=symbols/text]{menukeys}
% The default os is mac in symbol mode. 
% If we need text mode, just use \keys{Ctrl + Alt} instead of \keys{\ctrl + \Alt}
\usepackage{menukeys}
\ifnum \value{IsEnglishArticle}=0
% 用TimesNewRoman表现的TeX标志比较耐看,其他字体不好看。
\let\OldTeX\TeX
\renewcommand\TeX{{\TimesNewRoman\mdseries{\OldTeX\hspace{0.1em}}}}
\let\OldLaTeX\LaTeX
\renewcommand\LaTeX{{\TimesNewRoman\mdseries{\OldLaTeX\hspace{0.1em}}}} 
\let\OldLaTeXe\LaTeXe
\renewcommand\LaTeXe{{\TimesNewRoman\mdseries{\OldLaTeXe\hspace{0.1em}}}} 
\let\OldXeTeX\XeTeX
\renewcommand\XeTeX{{\TimesNewRoman\mdseries{\OldXeTeX\hspace{0.1em}}}}
\let\OldXeLaTeX\XeLaTeX
\renewcommand\XeLaTeX{{\TimesNewRoman\mdseries{\OldXeLaTeX\hspace{0.1em}}}}
\let\OldBibTeX\BibTeX
\renewcommand\BibTeX{{\TimesNewRoman\mdseries{\OldBibTeX\hspace{0.1em}}}}
\let\OldMF\MF
\renewcommand\MF{{\TimesNewRoman\mdseries{\OldMF\hspace{0.1em}}}}
\let\OldMP\MP
\renewcommand\MP{{\TimesNewRoman\mdseries{\OldMP\hspace{0.1em}}}}
\let\OldAmS\AmS
\renewcommand\AmS{{\TimesNewRoman\mdseries{\OldAmS\hspace{0.1em}}}}
\fi

% 危险警告标志
\newcommand*{\TakeFourierOrnament}[1]{{%
\fontencoding{U}\fontfamily{futs}\selectfont\char#1}}
\newcommand*{\Warning}{{
	\bfseries\Arial\larger\larger{
		\raisebox{0.5\depth}{\TakeFourierOrnament{66}}
		}
	}}
% 漂亮的键盘符号
% Handy switcher for menukeys. Switch from win to mac, from text to symbol, and vice versa.
\makeatletter
\def\SetMenuKeysToWin{\def\tw@mk@os{win}}
\def\SetMenuKeysToMac{\def\tw@mk@os{mac}}
\def\SetMacKeysText{\def\tw@mk@mackeys{text}}
\def\SetMacKeysSymbol{\def\tw@mk@mackeys{symbol}}
\makeatother

% 设置munukey的key风格
\renewcommand\delname{Del} % 让Delete键显示为"Del"而不是默认的"Del."
\let\OldKeys=\keys
\newcommand\Keys[1]{\hspace{0.5em}\OldKeys{#1}\hspace{0.5em}}
\renewmenumacro{\keys}[+]{shadowedroundedkeys} % Here, '+' means "Ctrl + Alt + Del"
\newcommand\ShadowedKeys[1]{\hspace{0.5em}\keys{#1}\hspace{0.5em}}

% 设置munukey的menu风格
\let\FirstOldMenu=\menu
\renewmenumacro{\menu}[>]{menus}
\newcommand\SimpleMenu[1]{\hspace{0.5em}\menu[>]{#1}\hspace{0.5em}}

% 设置munukey的menu风格
\let\SecondOldMenu=\FirstOldMenu
\renewmenumacro{\FirstOldMenu}[>]{roundedmenus}
\newcommand\RoundedMenu[1]{\hspace{0.5em}\FirstOldMenu[>]{#1}\hspace{0.5em}}

% 设置munukey的menu风格
\let\ThirdOldMenu=\SecondOldMenu
\renewmenumacro{\SecondOldMenu}[>]{angularmenus}
\newcommand\AngularMenu[1]{\hspace{0.5em}\SecondOldMenu[>]{#1}\hspace{0.5em}} % 漂亮的键盘符号
% 漂亮的文本段落线框和背景颜色,例如大段引用区域。
\definecolor{FrameBackgroundColor}{gray}{0.95}
\definecolor{FrameTitleBackgroundColor}{named}{lightgray}
\mdfsetup{
	backgroundcolor=FrameBackgroundColor,
	skipabove=1.5em,
	skipbelow=0em,
	leftmargin=2em,
	rightmargin=2em,
	innertopmargin=1em,
	innerbottommargin=1em,
	innerleftmargin=1em,
	innerrightmargin=1em,
	frametitleaboveskip=0.5em,
	frametitlebelowskip=0.5em,
	frametitlerule=true,
	frametitlealignment={\center},% default=\raggedleft
	frametitlebackgroundcolor=FrameTitleBackgroundColor,
	splittopskip=2\topsep% Useful when the frames are split on several pages.
}
%
% mdfsetup:
%
% Title的对齐方式有4种:左中右和微调。注意该包有一个bug:左右对齐的顺序搞反了。
% 居中:\mdfsetup{frametitlealignment={\center}}
% 微调:\mdfsetup{frametitlealignment={\hspace*{-2em}}} 默认情况下坐标0从缩进处开始计算
% 本来应该是右对齐,结果是左对齐。\mdfsetup{frametitlealignment={\raggedright}} 坐标0从未缩进处开始计算
% 本来应该是左对齐,结果是右对齐。\mdfsetup{frametitlealignment={\raggedleft}}
%
% 用法:\begin{mdframed}[frametitle={资源消耗非常小}],后面的可选参数设置标题。不设置就没有。
% % 漂亮的文本段落线框和背景颜色,例如大段引用区域。
% 技巧!楷书+斜体,中文楷书没有斜体,因此就是普通楷书,英文则以斜体显示。
% \AdobeKaiti\CourierNew{TextBody}含义:优先使用CourierNew,如果某个字符CourierNew无法显示,则用AdobeKaiti。
% 效果:英文字符可以用CourierNew显示,因此英文部分显示为CourierNew。但是CourierNew不认识中文,
% 于是中文就用AdobeKaiti输出。
\newenvironment{QuoteChinese}{\begin{mdframed}\begin{quoting}\kaiti\CourierNew}{\end{quoting}\end{mdframed}}
\newenvironment{QuoteEnglish}{\begin{mdframed}\begin{quoting}\TimesNewRoman\itshape{}}{\end{quoting}\end{mdframed}} % 漂亮的中英文段落引用
% 超链接格式
\hypersetup
{
	colorlinks=true,
	linkcolor=red,
	urlcolor=blue,
	citecolor=red,
	bookmarksopen=false,
	bookmarksnumbered=true,
	pdfborder=0 0 0
}

% 最佳脚注格式
\makeatletter
\newlength{\fnBreite}
\renewcommand{\@makefntext}[1]
{
	\settowidth{\fnBreite}{\footnotesize\@thefnmark.i}
	\protect\footnotesize\upshape
	\setlength{\@tempdima}{\columnwidth}\addtolength{\@tempdima}{-\fnBreite}
	\makebox[\fnBreite][l]{\@thefnmark \phantom{}}
	\parbox[t]{\@tempdima}{\everypar{\hspace*{1em}}\hspace*{-1em}\upshape\fangsong{#1}
}}
\makeatother
\renewcommand\thefootnote{\ding{\numexpr171+\value{footnote}}} % 定义脚注的圆圈符号
\renewcommand\footnotesep{1.5em} % 各条脚注段落之间的距离

\newcommand\Reference[1]{\textsuperscript{[\ref{#1}]}} % 设置脚注

\newcommand\RefLRS[1]{\hspace{0.5em}\ref{#1}\hspace{0.5em}} % 请参考 1.6 。
\newcommand\RefLS[1]{\hspace{0.5em}\ref{#1}} % 请参考 1.6。
\newcommand\RefRS[1]{\ref{#1}\hspace{0.5em}} % 请参考1.6 。章节 1.6显示:
\newcommand\Ref[1]{\ref{#1}} % 请参考1.6。

\newcommand\PageRefLRS[1]{\hspace{0.5em}\pageref{#1}\hspace{0.5em}} % 请参考第 3 页。
\newcommand\PageRefLS[1]{\hspace{0.5em}\pageref{#1}} % 请参考第 3页。
\newcommand\PageRefRS[1]{\pageref{#1}\hspace{0.5em}} % 请参考第3 页。
\newcommand\PageRef[1]{\pageref{#1}} % 请参考第3页。

\newcommand\UrlRefLRS[2]{\href{#1}{\hspace{0.5em}\bfseries{#2}\hspace{0.5em}}} % 网页超链接
\newcommand\UrlRefLS[2]{\href{#1}{\hspace{0.5em}\bfseries{#2}}} % 网页超链接
\newcommand\UrlRefRS[2]{\href{#1}{\bfseries{#2}\hspace{0.5em}}} % 网页超链接
\newcommand\UrlRef[2]{\href{#1}{\bfseries{#2}}} % 网页超链接

\newcommand\LinkRefLRS[2]{\hyperref[#1]{\hspace{0.5em}\bfseries{#2}\hspace{0.5em}}} % 文章内部超链接
\newcommand\LinkRefLS[2]{\hyperref[#1]{\hspace{0.5em}\bfseries{#2}}} % 文章内部超链接
\newcommand\LinkRefRS[2]{\hyperref[#1]{\bfseries{#2}\hspace{0.5em}}} % 文章内部超链接
\newcommand\LinkRef[2]{\hyperref[#1]{\bfseries{#2}}} % 文章内部超链接 % 引用、脚注、url超链接,文内超链接
\AtBeginDocument{% Code编号方案,很有学习价值
  \counterwithin*{lstlisting}{section}
  \renewcommand{\thelstlisting}{%
    \ifnum\value{subsection}=0
      \thesection-\arabic{lstlisting}%    
    \fi
  }%
} % e.g. Code 4-1: Hello.cpp

% Code编号方案,很有学习价值
%\AtBeginDocument{% Code编号方案,很有学习价值
%  \counterwithin*{lstlisting}{section}
%  \counterwithin*{lstlisting}{subsection}
%  \counterwithin*{lstlisting}{subsubsection}
%  \renewcommand{\thelstlisting}{%
%    \ifnum\value{subsection}=0
%      \thesection.\arabic{lstlisting}%
%    \else
%      \ifnum\value{subsubsection}=0
%        \thesubsection.\arabic{lstlisting}%
%      \else
%        \thesubsubsection.\arabic{lstlisting}%
%      \fi
%    \fi
%  }%
%} % Code 4.1: Hello.cpp

% 源代码格式,Label和标题必须同时出现
\renewcommand*{\lstlistingname}{Code}
\DeclareCaptionFormat{listing}{\centering\AdobeHeiti\footnotesize{\bfseries{#1#2}\mdseries{#3}}} % Label|Separator|Title
\captionsetup[lstlisting]{format=listing}
\lstset
{
	backgroundcolor=\color{lightgray},
	basicstyle=\ttfamily\CodeFont\mdseries,
	keywordstyle=\bfseries,
	commentstyle=\rmfamily\itshape\TimesNewRoman,
	columns=fixed,
	keepspaces=true,
	showstringspaces=false,
	%stringstyle=\ttfamily\AdobeHeiti\mdseries,
	xleftmargin=2em,
	numbers=none,
	numberstyle=\AdobeHeiti\mdseries\footnotesize,
	numbersep=1em,
	tabsize=4,
	showtabs=false,
}
\lstnewenvironment{Listing}[1]{\lstset{language={#1},numbers=none}}{}
\lstnewenvironment{ListingWithNumbers}[1]{\lstset{language={#1},numbers=left}}{}
\lstnewenvironment{ListingWithTitleNoNumbers}[2]{\lstset{language={#1},title={\AdobeHeiti\footnotesize #2},numbers=none}}{}
\lstnewenvironment{ListingWithTitleWithNumbers}[2]{\lstset{language={#1},title={\AdobeHeiti\footnotesize #2},numbers=left}}{}
\lstnewenvironment{ListingWithCaptionNoNumbers}[2]{\lstset{language={#1},caption={\AdobeHeiti\footnotesize #2},numbers=none}}{}
\lstnewenvironment{ListingWithCaptionWithNumbers}[2]{\lstset{language={#1},caption={\AdobeHeiti\footnotesize #2},numbers=left}}{} % 原文抄录,例如源代码
% 设置图表显示格式
\renewcommand{\thefigure}{\thesection-\arabic{figure}}
\renewcommand{\thetable}{\thesection-\arabic{table}}
\DeclareCaptionLabelFormat{FigureLabel}{\sffamily\bfseries\footnotesize{#1\hspace{0.5em}#2}}
\DeclareCaptionLabelFormat{TableLabel}{\sffamily\bfseries\footnotesize{\raisebox{-1ex}{#1\hspace{0.5em}#2}}}

\ifnum \value{IsEnglishArticle}>0
    \captionsetup[figure]{labelformat=FigureLabel,skip=1em,labelsep=none,name={Figure}}
	\captionsetup[table]{labelformat=TableLabel,skip=1em,labelsep=none,name={Table}}
\else
	\captionsetup[figure]{labelformat=FigureLabel,skip=1em,labelsep=none,name={图}}
	\captionsetup[table]{labelformat=TableLabel,skip=1em,labelsep=none,name={表}}
\fi

% 唯一的参数代表Caption。Caption为空的情况下,不出现冒号
% \ifstrempty 是非常棒的条件测试工具!
\NewDocumentEnvironment{Figure}{m}
	{\begin{figure}[htp]\centering}
	{
		\ifstrempty{#1}
			{\caption{#1}}
			{\caption{\footnotesize\hspace{1em}#1}}
	\end{figure}
	}

% 唯一的参数代表顶部Caption。Caption不可为空。
\NewDocumentEnvironment{Table}{m}
	{\begin{table}[!h]
		\centering
		\captionsetup{font={bf,sf},skip=0.5ex}
		\caption*{{\large{#1}}}
	}
	{
		\caption{}
	\end{table}
	}
	
% 设置三线式表格
\setlength\heavyrulewidth{1.5pt} % 粗线宽度
\setlength\aboverulesep{0.5ex} % 线前距离
\setlength\belowrulesep{1ex} % 线后距离

% 使每一个Section的计数器自动清零,这样图表的编号就以section为编号空间。
\counterwithin{figure}{section}
\counterwithin{table}{section} % 图表
% 列表格式 %%%%%%%%%%%%%%%%%%%%%%%%%%%%%%%%%%%%%%%%%
%                                                  % 
% labelindent is for label only;                   %
% leftmargin is for the item text only;            %
% we can override the settings in place if needed. %
%                                                  %
% It is LaTeX's rule that if the optional argument %
% is not present, then #1 takes on the default     %
% value.                                           % 
%                                                  % 
%%%%%%%%%%%%%%%%%%%%%%%%%%%%%%%%%%%%%%%%%%%%%%%%%%%%
\newenvironment{Enumerate}{\begin{enumerate}[leftmargin=4em]}{\end{enumerate}}
\newenvironment{Itemize}{\begin{itemize}[leftmargin=4em]}{\end{itemize}}
\newenvironment{Description}[1][6em]{ % Only 1 parameter, and the default value is 6em
\begin{description}[style=multiline,labelindent=2em,leftmargin=#1,rightmargin=2em]}
{\end{description}}
\renewcommand\labelenumi{\TimesNewRoman\mdseries\rmfamily{(\theenumi)}} % (1), (2), (3), ... % 有序列表、无序列表、描述列表
% 确保章节自动编号的字体和章节标题字体看起来一样大(这里的调整不影响目录)
\titleformat{\section}[hang]{\huge\AdobeHeiti\bfseries}{\thesection}{1em}{}
\titleformat{\subsection}[hang]{\LARGE\AdobeHeiti\bfseries}{\thesubsection}{1em}{}
\titleformat{\subsubsection}[hang]{\Large\AdobeHeiti\bfseries}{\thesubsubsection}{1em}{}

% 各级标题的边距和行距
\titlespacing*{\section}{0pt}{1em}{0em} % 含义:左边距的<增加值>为0,段前距为1em,段后距为0,末尾的可选项为右边距,忽略取默认值。
\titlespacing*{\subsection}{0pt}{1em}{0em}
\titlespacing*{\subsubsection}{0pt}{1em}{0em} % 标题格式控制

% 默认段距
\setlength\parskip{1em}

% em is the TeX command and emph is its LaTeX equivalent. Therefore, you should use the emph when using LaTeX.
\renewcommand\emph[1]{{\sffamily\bfseries{\hspace{0em}#1\hspace{0em}}}}