% 漂亮的文本段落线框和背景颜色,例如大段引用区域。
\definecolor{FrameBackgroundColor}{gray}{0.95}
\definecolor{FrameTitleBackgroundColor}{named}{lightgray}
\mdfsetup{
	backgroundcolor=FrameBackgroundColor,
	skipabove=1.5em,
	skipbelow=0em,
	leftmargin=2em,
	rightmargin=2em,
	innertopmargin=1em,
	innerbottommargin=1em,
	innerleftmargin=1em,
	innerrightmargin=1em,
	frametitleaboveskip=0.5em,
	frametitlebelowskip=0.5em,
	frametitlerule=true,
	frametitlealignment={\center},% default=\raggedleft
	frametitlebackgroundcolor=FrameTitleBackgroundColor,
	splittopskip=2\topsep% Useful when the frames are split on several pages.
}
%
% mdfsetup:
%
% Title的对齐方式有4种:左中右和微调。注意该包有一个bug:左右对齐的顺序搞反了。
% 居中:\mdfsetup{frametitlealignment={\center}}
% 微调:\mdfsetup{frametitlealignment={\hspace*{-2em}}} 默认情况下坐标0从缩进处开始计算
% 本来应该是右对齐,结果是左对齐。\mdfsetup{frametitlealignment={\raggedright}} 坐标0从未缩进处开始计算
% 本来应该是左对齐,结果是右对齐。\mdfsetup{frametitlealignment={\raggedleft}}
%
% 用法:\begin{mdframed}[frametitle={资源消耗非常小}],后面的可选参数设置标题。不设置就没有。
%