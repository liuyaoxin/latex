% 设置目录中的标题字体。默认情况很好,故无需启用。
%\renewcommand{\cftsecfont}{\YaHei\bfseries}
%\renewcommand{\cftsubsecfont}{\YaHei}
%\renewcommand{\cftsubsubsecfont}{\YaHei}

\renewcommand{\cftsecnumwidth}{2em}
\renewcommand{\cftsubsecnumwidth}{3em}
\renewcommand{\cftsubsubsecnumwidth}{3em}

\renewcommand{\cftsecindent}{0em}
\renewcommand{\cftsubsecindent}{2em}
\renewcommand{\cftsubsubsecindent}{5em}
\renewcommand{\cftbeforesecskip}{1em}
\renewcommand{\cftbeforesubsecskip}{1em}
\renewcommand{\cftbeforesubsubsecskip}{1em}

% 单独调整目录项中的数字编号字体。默认情况很好,故无需启用。
%\renewcommand{\cftsecpresnum}{\bfseries\rmfamily}
%\renewcommand{\cftsubsecpresnum}{\mdseries\rmfamily}
%\renewcommand{\cftsubsubsecpresnum}{\mdseries\rmfamily}

% 设置目录二字后面的行距
\setlength\cftaftertoctitleskip{2em}

\ifnum \value{IsEnglishArticle}>0
    \renewcommand{\contentsname}{\centerline{\large\bfseries\Arial{Table of Contents}}}
	\pdfbookmark[1]{Table of Contents}{contents}
	\tableofcontents
\else
	\renewcommand{\contentsname}{\centerline{\large\heiti{目 \quad 录}}}
	\pdfbookmark[1]{目录}{contents}
	\tableofcontents
\fi